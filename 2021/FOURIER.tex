\begin{enumerate}[label=\thechapter.\arabic*,ref=\thechapter.\theenumi]
\item Consider the signals $x\brak{n}$=$2^{n-1} u\brak{ -n+2}$ and $y\brak{n}$=$2^{-n+2}u\brak{ n+1}$, where $u\brak{n}$ is the unit step sequence. Let $X\brak{e^{j\omega}}$ and $Y\brak{e^{j\omega}}$ be the discrete-time Fourier of $x\brak{n}$ and $y\brak{n}$,respectively. The value of the integral $\frac{1}{2\pi}\int_{0}^{2\pi} X\brak{e^{j\omega}} Y\brak{e^{-j\omega}} d \omega$
(rounded off to one decimal place) is \underline{{\hspace{1.5in}}}\\
\hfill{(GATE EC 41 2021)}\\
\solution
\input{2021/EC/41/gate.tex}
\pagebreak
\item Given that $\mathcal{S}$ is the unit circle in the counter clock-wise direction with its centre at origin, the integral
        $\oint \brak{\frac{z^3}{4z-\jmath}}dz=\rule{1cm}{0.15mm}$
 (round off to theree decimal places)
 \hfill{(GATE 2022 AE)}\\
 \solution\\
 \iffalse
\let\negmedspace\undefined
\let\negthickspace\undefined
\documentclass[journal,12pt,twocolumn]{IEEEtran}
\usepackage{cite}
\usepackage{amsmath,amssymb,amsfonts,amsthm}
\usepackage{algorithmic}
\usepackage{graphicx}
\usepackage{textcomp}
\usepackage{xcolor}
\usepackage{txfonts}
\usepackage{listings}
\usepackage{enumitem}
\usepackage{mathtools}
\usepackage{gensymb}
\usepackage{comment}
\usepackage[breaklinks=true]{hyperref}
\usepackage{tkz-euclide} 
\usepackage{listings}
\usepackage{gvv}                                        
\def\inputGnumericTable{}                                 
\usepackage[latin1]{inputenc}                                
\usepackage{color}                                            
\usepackage{array}                                            
\usepackage{longtable}                                       
\usepackage{calc}                                             
\usepackage{multirow}                                         
\usepackage{hhline}                                           
\usepackage{ifthen}                                           
\usepackage{lscape}
\usepackage[center]{caption} % center the captions to figure

\newtheorem{theorem}{Theorem}[section]
\newtheorem{problem}{Problem}
\newtheorem{proposition}{Proposition}[section]
\newtheorem{lemma}{Lemma}[section]
\newtheorem{corollary}[theorem]{Corollary}
\newtheorem{example}{Example}[section]
\newtheorem{definition}[problem]{Definition}
\newcommand{\BEQA}{\begin{eqnarray}}
\newcommand{\EEQA}{\end{eqnarray}}
\newcommand{\define}{\stackrel{\triangle}{=}}
\theoremstyle{remark}
\newtheorem{rem}{Remark}
\begin{document}

\newcolumntype{M}[1]{>{\centering\arraybackslash}m{#1}}
\newcolumntype{N}{@{}m{0pt}@{}}

\bibliographystyle{IEEEtran}
\vspace{3cm}

\title{GATE 2021 ME 3Q} 
\author{ee23btech11223 - Soham Prabhakar More% <-this % stops a space
}
\maketitle
\newpage
\bigskip

\renewcommand{\thefigure}{\theenumi}
\renewcommand{\thetable}{\theenumi}

\bibliographystyle{IEEEtran}

\textbf{Question:} The Dirac-delta function $\brak{\delta\brak{t - t_0}}$ for $t, t_0 \in \Re$, has the following property
\begin{align}
    \int_{a}^{b}\phi\brak{t}\delta\brak{t - t_0}dt = 
    \begin{cases}
        \phi\brak{t_0}\quad a < t_0 < b\\
        0 \quad\quad otherwise
    \end{cases} \label{eq:2022.ME.3.1}
\end{align}

The Laplace Transform of the Dirac-delta function $\delta\brak{t - a}$ for $a > 0; \mathcal{L}\brak{\delta\brak{t - a}} = F\brak{s}$ is

\hfill{(GATE 2021 ME 3Q)}

\solution
\fi
\begin{table}[ht]
    \renewcommand\thetable{1}
\begin{tabular}{|c|c|}
    \hline 
    \textbf{Parameter}&\textbf{Description} \\
    \hline
    $F\brak{s}$ & Laplace transform of $\delta\brak{t - a}$ \\
    \hline
    $G\brak{f}$ & Fourier transform of $\delta\brak{t - a}$ \\
    \hline
    $H\brak{t}$ & Fourier transform of a function with period $T$ \\
    \hline
    $w_T\brak{t}$ & Delta Comb, $\sum_{k = -\infty}^{\infty}\delta\brak{t - kT}$ \\
    \hline
    $W_T\brak{t}$ & Fourier transform of $w_T\brak{t}$ \\
    \hline
\end{tabular}

\caption{Table of parameters}
\label{Table:2022.ME.3.1}


\end{table} \\

By \eqref{eq:2022.ME.3.1} and $a > 0$,
\begin{align}
    F\brak{s} &= \int_{0}^{\infty}\delta\brak{t - a}e^{-st}dt \\
    \therefore F\brak{s} &= e^{-as}
\end{align}

The fourier transform,
\begin{align}
    G\brak{f} &= \int_{-\infty}^{\infty}\delta\brak{t - a}e^{-2\pi jft}dt \\
    \therefore G\brak{f} &= e^{-j2\pi fa}
\end{align}
For a periodic signal the fourier transform is defined as:
\begin{align}
    H\brak{f} &= \sum_{k = -\infty}^{\infty}c_k\delta\brak{f - \frac{k}{T}}
\end{align}
where $c_k$ are the fourier series coefficients and $T$ is the period. Thus,
\begin{align}
    W_T\brak{f} &= \sum_{k = -\infty}^{\infty}c_k\delta\brak{f - \frac{k}{T}} \\
    c_k &= \frac{1}{T}\int_{-\frac{T}{2}}^{\frac{T}{2}}w_T\brak{t}e^{-j2\pi \frac{k}{T}f}dt \\
    c_k &= \frac{1}{T}\int_{-\frac{T}{2}}^{\frac{T}{2}}\brak{\sum_{k = -\infty}^{\infty}\delta\brak{t - kT}}e^{-j2\pi \frac{k}{T}f}dt \\
\end{align}
\begin{align}
    c_k &= \frac{1}{T}\sum_{k = -\infty}^{\infty}\int_{-\frac{T}{2}}^{\frac{T}{2}}\delta\brak{t - kT}e^{-j2\pi \frac{k}{T}f}dt \\
    c_k &= \frac{1}{T} \\
    W_T\brak{f} &= \frac{1}{T}\sum_{k = -\infty}^{\infty}\delta\brak{f - \frac{k}{T}} \\
    \therefore W_T\brak{f} &= \frac{1}{T}w_{\frac{1}{T}}\brak{f}
\end{align}
Thus, the fourier transform of impulse train is another impulse train.
%\begin{align}
%    f\brak{t} &\system{F} H\brak{f} \\
%    f\brak{t + T} &\system{F} e^{j2\pi fT}H\brak{f} \\
%    \because e^{j2\pi fT}H\brak{f} &= H\brak{f} \\
%    H\brak{f}\brak{1 - e^{j2\pi fT}} &= 0
%\end{align}
%Thus, $H\brak{f}$ is zero everywhere except at $f = \frac{n}{T}, n \in Z$
%\begin{align}
%    \therefore H\brak{f} &= \sum_{k = -\infty}^{\infty}c_k\delta\brak{f - \frac{k}{T}} \\
%    \because \sum_{k = -\infty}^{\infty}c_ke^{-j2\pi f\frac{k}{T}} &\system{F} H\brak{f}
%\end{align}
%$c_k$ are the fourier series coefficents of $h\brak{t}$,
%\begin{align}
%    c_k = \int_{-\frac{T}{2}}^{\frac{T}{2}}
%\end{align}
%\end{document}


\item Consider the signals \(x[n] = 2^{n-1} u[-n+2]\) and \(y[n] = 2^{-n+2} u[n+1]\), where \(u[n]\) is the unit step sequence. Let \(X(e^{j\omega})\) and \(Y(e^{j\omega})\) be the discrete-time Fourier transform of \(x[n]\) and \(y[n]\), respectively. The value of the integral
\[
\frac{1}{2\pi} \int_{0}^{2\pi} X(e^{j\omega}) Y(e^{-j\omega}) d\omega
\]
(rounded off to one decimal place) is.\\
\hfill{GATE 2021 EC 41 Q}
\solution
\input{2021/EC/41Q/g.tex}
\pagebreak
\item Consider a continuous-time signal $x\brak{t}$ \,defined by $x\brak{t}=0$\,for $\abs{t}>1$, and $x\brak{t}=1-\abs{t}$ for $\abs{t}\le 1$. Let the Fourier transform of $x\brak{t}$ be defined as $X\brak{\omega}=\int_{-\infty}^{\infty}x\brak{t}e^{-j\omega t} dt$. The maximum magnitude of $X\brak{\omega}$ is $\hbox to 4em{\thinspace\hrulefill\thinspace}$.
\hfill{(GATE 2021 EE 43)}\\
\solution
\iffalse
\let\negmedspace\undefined
\let\negthickspace\undefined
\documentclass[journal,12pt,twocolumn]{IEEEtran}
\usepackage{cite}
\usepackage{amsmath,amssymb,amsfonts,amsthm}
\usepackage{algorithmic}
\usepackage{graphicx}
\usepackage{textcomp}
\usepackage{xcolor}
\usepackage{txfonts}
\usepackage{listings}
\usepackage{enumitem}
\usepackage{mathtools}
\usepackage{gensymb}
\usepackage{comment}
\usepackage[breaklinks=true]{hyperref}
\usepackage{tkz-euclide} 
\usepackage{listings}
\usepackage{gvv}                                        
\def\inputGnumericTable{}                                 
\usepackage[latin1]{inputenc}                                
\usepackage{color}                                            
\usepackage{array}                                            
\usepackage{longtable}                                       
\usepackage{calc}                                             
\usepackage{multirow}                                         
\usepackage{hhline}                                           
\usepackage{ifthen}                                           
\usepackage{lscape}
\usepackage[center]{caption} % center the captions to figure

\newtheorem{theorem}{Theorem}[section]
\newtheorem{problem}{Problem}
\newtheorem{proposition}{Proposition}[section]
\newtheorem{lemma}{Lemma}[section]
\newtheorem{corollary}[theorem]{Corollary}
\newtheorem{example}{Example}[section]
\newtheorem{definition}[problem]{Definition}
\newcommand{\BEQA}{\begin{eqnarray}}
\newcommand{\EEQA}{\end{eqnarray}}
\newcommand{\define}{\stackrel{\triangle}{=}}
\theoremstyle{remark}
\newtheorem{rem}{Remark}
\begin{document}

\newcolumntype{M}[1]{>{\centering\arraybackslash}m{#1}}
\newcolumntype{N}{@{}m{0pt}@{}}

\bibliographystyle{IEEEtran}
\vspace{3cm}

\title{GATE 2021 ME 3Q} 
\author{ee23btech11223 - Soham Prabhakar More% <-this % stops a space
}
\maketitle
\newpage
\bigskip

\renewcommand{\thefigure}{\theenumi}
\renewcommand{\thetable}{\theenumi}

\bibliographystyle{IEEEtran}

\textbf{Question:} The Dirac-delta function $\brak{\delta\brak{t - t_0}}$ for $t, t_0 \in \Re$, has the following property
\begin{align}
    \int_{a}^{b}\phi\brak{t}\delta\brak{t - t_0}dt = 
    \begin{cases}
        \phi\brak{t_0}\quad a < t_0 < b\\
        0 \quad\quad otherwise
    \end{cases} \label{eq:2022.ME.3.1}
\end{align}

The Laplace Transform of the Dirac-delta function $\delta\brak{t - a}$ for $a > 0; \mathcal{L}\brak{\delta\brak{t - a}} = F\brak{s}$ is

\hfill{(GATE 2021 ME 3Q)}

\solution
\fi
\begin{table}[ht]
    \renewcommand\thetable{1}
\begin{tabular}{|c|c|}
    \hline 
    \textbf{Parameter}&\textbf{Description} \\
    \hline
    $F\brak{s}$ & Laplace transform of $\delta\brak{t - a}$ \\
    \hline
    $G\brak{f}$ & Fourier transform of $\delta\brak{t - a}$ \\
    \hline
    $H\brak{t}$ & Fourier transform of a function with period $T$ \\
    \hline
    $w_T\brak{t}$ & Delta Comb, $\sum_{k = -\infty}^{\infty}\delta\brak{t - kT}$ \\
    \hline
    $W_T\brak{t}$ & Fourier transform of $w_T\brak{t}$ \\
    \hline
\end{tabular}

\caption{Table of parameters}
\label{Table:2022.ME.3.1}


\end{table} \\

By \eqref{eq:2022.ME.3.1} and $a > 0$,
\begin{align}
    F\brak{s} &= \int_{0}^{\infty}\delta\brak{t - a}e^{-st}dt \\
    \therefore F\brak{s} &= e^{-as}
\end{align}

The fourier transform,
\begin{align}
    G\brak{f} &= \int_{-\infty}^{\infty}\delta\brak{t - a}e^{-2\pi jft}dt \\
    \therefore G\brak{f} &= e^{-j2\pi fa}
\end{align}
For a periodic signal the fourier transform is defined as:
\begin{align}
    H\brak{f} &= \sum_{k = -\infty}^{\infty}c_k\delta\brak{f - \frac{k}{T}}
\end{align}
where $c_k$ are the fourier series coefficients and $T$ is the period. Thus,
\begin{align}
    W_T\brak{f} &= \sum_{k = -\infty}^{\infty}c_k\delta\brak{f - \frac{k}{T}} \\
    c_k &= \frac{1}{T}\int_{-\frac{T}{2}}^{\frac{T}{2}}w_T\brak{t}e^{-j2\pi \frac{k}{T}f}dt \\
    c_k &= \frac{1}{T}\int_{-\frac{T}{2}}^{\frac{T}{2}}\brak{\sum_{k = -\infty}^{\infty}\delta\brak{t - kT}}e^{-j2\pi \frac{k}{T}f}dt \\
\end{align}
\begin{align}
    c_k &= \frac{1}{T}\sum_{k = -\infty}^{\infty}\int_{-\frac{T}{2}}^{\frac{T}{2}}\delta\brak{t - kT}e^{-j2\pi \frac{k}{T}f}dt \\
    c_k &= \frac{1}{T} \\
    W_T\brak{f} &= \frac{1}{T}\sum_{k = -\infty}^{\infty}\delta\brak{f - \frac{k}{T}} \\
    \therefore W_T\brak{f} &= \frac{1}{T}w_{\frac{1}{T}}\brak{f}
\end{align}
Thus, the fourier transform of impulse train is another impulse train.
%\begin{align}
%    f\brak{t} &\system{F} H\brak{f} \\
%    f\brak{t + T} &\system{F} e^{j2\pi fT}H\brak{f} \\
%    \because e^{j2\pi fT}H\brak{f} &= H\brak{f} \\
%    H\brak{f}\brak{1 - e^{j2\pi fT}} &= 0
%\end{align}
%Thus, $H\brak{f}$ is zero everywhere except at $f = \frac{n}{T}, n \in Z$
%\begin{align}
%    \therefore H\brak{f} &= \sum_{k = -\infty}^{\infty}c_k\delta\brak{f - \frac{k}{T}} \\
%    \because \sum_{k = -\infty}^{\infty}c_ke^{-j2\pi f\frac{k}{T}} &\system{F} H\brak{f}
%\end{align}
%$c_k$ are the fourier series coefficents of $h\brak{t}$,
%\begin{align}
%    c_k = \int_{-\frac{T}{2}}^{\frac{T}{2}}
%\end{align}
%\end{document}

\pagebreak
\item Let $f(t)$ be an even function, i.e.$f(-t) = f(t)$ for all t.Let the Fourier transform of $f(t)$ be defined as $F(\omega) = \int_{-\infty}^{\infty} f(t) e^{-j \omega t} \, dt $ . Suppose $\dfrac{dF(\omega)}{d \omega} = -\omega F(\omega)$ for all $\omega$ , and $F(0) = 1$ . Then


\begin{enumerate}[label = (\Alph*)]
\item $f(0) < 1 $\\
\item  $f(0) > 1 $\\
\item  $f(0) = 1 $\\
\item   $f(0) = 0 $\\
\end{enumerate} \hfill{(GATE EE 2021)}\\
\solution
\input{2021/EE/32/g.tex}
\pagebreak
\item The exponential Fourier series representation of a continous-time periodic signal x\brak{t} is defined as\\
\begin{center}
$x\brak{t}=\sum\limits_{k=-\infty}^{\infty}a_ke^{jk\omega_0t}$\\
\end{center}
where $\omega_0$ is the fundamental angular frequency of x\brak{t} and the coefficients of the series are $a_k$.The following information is given about x\brak{t} and $a_k$\\
I. x\brak{t} is real and even,having a fundamental period of 6\\
II. The average value of x\brak{t} is 2.\\
III.\begin{align}
 a_k= \begin{cases} 
      k, & 1 \leq k \leq 3 \\
      0, &  k > 3 
   \end{cases}\\
   \end{align}
The average power of the signal x\brak{t} (rounded off to one decimal place) is \underline{\hspace{1cm}}. \\
\hfill(GATE EC 2021)\\
\solution\\
\input{2021/EC/39/10.tex}
\pagebreak
\item Let $X\brak{j \omega}$ denotes the Fourier transform of $x\brak{t}$. If 
\begin{align}
X\brak{j \omega} =& 10e^{-j\pi f \: \brak{\dfrac{sin\brak{\pi f}}{\pi f}}}
\end{align} 
then $ \dfrac{1}{2\pi} \int_{-\infty }^{\infty} X\brak{j \omega} d\omega = \rule{1cm}{0.15mm}$ .  (where $\omega$ = $2\pi f$)\\
\begin{enumerate}[label = \brak{\Alph*}]
\item 10$\pi$ \\
\item 100 \\
\item 10 \\
\item 20$\pi$ 
\end{enumerate}
\hfill GATE 2021\\
\solution
\input{2021/BM/5/bm.tex}
\item Consider the sequence $ x_n = 0.5x_{n-1} + 1 , n = 1,2,...$ with $x_0 = 0$ Then $ \lim_{n \to \infty} x_n$ is :
\begin{enumerate}
\item[A]0
\item[B]1
\item[C]2
\item[D]$\infty$
\end{enumerate}
\hfill {GATE 2021 IN}\\
\solution
\input{2021/IN/2/in02.tex}
\pagebreak
\item The Dirac-delta function $\brak{\delta\brak{t - t_0}}$ for $t, t_0 \in \Re$, has the following property
\begin{align}
    \int_{a}^{b}\phi\brak{t}\delta\brak{t - t_0}dt = 
    \begin{cases}
        \phi\brak{t_0}\quad a < t_0 < b\\
        0 \quad\quad otherwise
    \end{cases} \label{eq:2022.ME.3.1}
\end{align}

The Laplace Transform of the Dirac-delta function $\delta\brak{t - a}$ for $a > 0; \mathcal{L}\brak{\delta\brak{t - a}} = F\brak{s}$ is

\hfill{(GATE 2021 ME 3Q)} \\
\solution
\iffalse
\let\negmedspace\undefined
\let\negthickspace\undefined
\documentclass[journal,12pt,twocolumn]{IEEEtran}
\usepackage{cite}
\usepackage{amsmath,amssymb,amsfonts,amsthm}
\usepackage{algorithmic}
\usepackage{graphicx}
\usepackage{textcomp}
\usepackage{xcolor}
\usepackage{txfonts}
\usepackage{listings}
\usepackage{enumitem}
\usepackage{mathtools}
\usepackage{gensymb}
\usepackage{comment}
\usepackage[breaklinks=true]{hyperref}
\usepackage{tkz-euclide} 
\usepackage{listings}
\usepackage{gvv}                                        
\def\inputGnumericTable{}                                 
\usepackage[latin1]{inputenc}                                
\usepackage{color}                                            
\usepackage{array}                                            
\usepackage{longtable}                                       
\usepackage{calc}                                             
\usepackage{multirow}                                         
\usepackage{hhline}                                           
\usepackage{ifthen}                                           
\usepackage{lscape}
\usepackage[center]{caption} % center the captions to figure

\newtheorem{theorem}{Theorem}[section]
\newtheorem{problem}{Problem}
\newtheorem{proposition}{Proposition}[section]
\newtheorem{lemma}{Lemma}[section]
\newtheorem{corollary}[theorem]{Corollary}
\newtheorem{example}{Example}[section]
\newtheorem{definition}[problem]{Definition}
\newcommand{\BEQA}{\begin{eqnarray}}
\newcommand{\EEQA}{\end{eqnarray}}
\newcommand{\define}{\stackrel{\triangle}{=}}
\theoremstyle{remark}
\newtheorem{rem}{Remark}
\begin{document}

\newcolumntype{M}[1]{>{\centering\arraybackslash}m{#1}}
\newcolumntype{N}{@{}m{0pt}@{}}

\bibliographystyle{IEEEtran}
\vspace{3cm}

\title{GATE 2021 ME 3Q} 
\author{ee23btech11223 - Soham Prabhakar More% <-this % stops a space
}
\maketitle
\newpage
\bigskip

\renewcommand{\thefigure}{\theenumi}
\renewcommand{\thetable}{\theenumi}

\bibliographystyle{IEEEtran}

\textbf{Question:} The Dirac-delta function $\brak{\delta\brak{t - t_0}}$ for $t, t_0 \in \Re$, has the following property
\begin{align}
    \int_{a}^{b}\phi\brak{t}\delta\brak{t - t_0}dt = 
    \begin{cases}
        \phi\brak{t_0}\quad a < t_0 < b\\
        0 \quad\quad otherwise
    \end{cases} \label{eq:2022.ME.3.1}
\end{align}

The Laplace Transform of the Dirac-delta function $\delta\brak{t - a}$ for $a > 0; \mathcal{L}\brak{\delta\brak{t - a}} = F\brak{s}$ is

\hfill{(GATE 2021 ME 3Q)}

\solution
\fi
\begin{table}[ht]
    \renewcommand\thetable{1}
\begin{tabular}{|c|c|}
    \hline 
    \textbf{Parameter}&\textbf{Description} \\
    \hline
    $F\brak{s}$ & Laplace transform of $\delta\brak{t - a}$ \\
    \hline
    $G\brak{f}$ & Fourier transform of $\delta\brak{t - a}$ \\
    \hline
    $H\brak{t}$ & Fourier transform of a function with period $T$ \\
    \hline
    $w_T\brak{t}$ & Delta Comb, $\sum_{k = -\infty}^{\infty}\delta\brak{t - kT}$ \\
    \hline
    $W_T\brak{t}$ & Fourier transform of $w_T\brak{t}$ \\
    \hline
\end{tabular}

\caption{Table of parameters}
\label{Table:2022.ME.3.1}


\end{table} \\

By \eqref{eq:2022.ME.3.1} and $a > 0$,
\begin{align}
    F\brak{s} &= \int_{0}^{\infty}\delta\brak{t - a}e^{-st}dt \\
    \therefore F\brak{s} &= e^{-as}
\end{align}

The fourier transform,
\begin{align}
    G\brak{f} &= \int_{-\infty}^{\infty}\delta\brak{t - a}e^{-2\pi jft}dt \\
    \therefore G\brak{f} &= e^{-j2\pi fa}
\end{align}
For a periodic signal the fourier transform is defined as:
\begin{align}
    H\brak{f} &= \sum_{k = -\infty}^{\infty}c_k\delta\brak{f - \frac{k}{T}}
\end{align}
where $c_k$ are the fourier series coefficients and $T$ is the period. Thus,
\begin{align}
    W_T\brak{f} &= \sum_{k = -\infty}^{\infty}c_k\delta\brak{f - \frac{k}{T}} \\
    c_k &= \frac{1}{T}\int_{-\frac{T}{2}}^{\frac{T}{2}}w_T\brak{t}e^{-j2\pi \frac{k}{T}f}dt \\
    c_k &= \frac{1}{T}\int_{-\frac{T}{2}}^{\frac{T}{2}}\brak{\sum_{k = -\infty}^{\infty}\delta\brak{t - kT}}e^{-j2\pi \frac{k}{T}f}dt \\
\end{align}
\begin{align}
    c_k &= \frac{1}{T}\sum_{k = -\infty}^{\infty}\int_{-\frac{T}{2}}^{\frac{T}{2}}\delta\brak{t - kT}e^{-j2\pi \frac{k}{T}f}dt \\
    c_k &= \frac{1}{T} \\
    W_T\brak{f} &= \frac{1}{T}\sum_{k = -\infty}^{\infty}\delta\brak{f - \frac{k}{T}} \\
    \therefore W_T\brak{f} &= \frac{1}{T}w_{\frac{1}{T}}\brak{f}
\end{align}
Thus, the fourier transform of impulse train is another impulse train.
%\begin{align}
%    f\brak{t} &\system{F} H\brak{f} \\
%    f\brak{t + T} &\system{F} e^{j2\pi fT}H\brak{f} \\
%    \because e^{j2\pi fT}H\brak{f} &= H\brak{f} \\
%    H\brak{f}\brak{1 - e^{j2\pi fT}} &= 0
%\end{align}
%Thus, $H\brak{f}$ is zero everywhere except at $f = \frac{n}{T}, n \in Z$
%\begin{align}
%    \therefore H\brak{f} &= \sum_{k = -\infty}^{\infty}c_k\delta\brak{f - \frac{k}{T}} \\
%    \because \sum_{k = -\infty}^{\infty}c_ke^{-j2\pi f\frac{k}{T}} &\system{F} H\brak{f}
%\end{align}
%$c_k$ are the fourier series coefficents of $h\brak{t}$,
%\begin{align}
%    c_k = \int_{-\frac{T}{2}}^{\frac{T}{2}}
%\end{align}
%\end{document}

\end{enumerate}
